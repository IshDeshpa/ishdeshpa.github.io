\documentclass[10pt]{article}
\usepackage{geometry}
\usepackage{enumitem}
\usepackage{helvet}
\usepackage[T1]{fontenc}
\usepackage{xcolor}
\usepackage{hyperref}

\geometry{
    letterpaper,
    left=0.5in,
    right=0.5in,
    top=0.5in,
    bottom=0.5in
}

\pagenumbering{gobble}
\input{glyphtounicode}
\pdfgentounicode=1 

\hypersetup{
    pdfauthor={Ishan Deshpande},
    pdftitle={Resume},
    pdfsubject={Resume},
    pdfkeywords={resume, Ishan Deshpande, electrical engineering, computer engineering}
}

\renewcommand{\familydefault}{\sfdefault}

\newcommand{\sectionHeader}[1]{%
    \vspace{-1.25\baselineskip}
    \section*{\large \MakeUppercase{#1}}
    \vspace{-1.5\baselineskip}
    \color{teal}
    \rule{\textwidth}{1.5pt} % Add a horizontal line
    \color{black}
    \vspace{-1.25\baselineskip}
    \pdfbookmark[1]{#1}{#1}
}

\newcommand{\role}[3]{
    \textit{#1} \hfill \textit{#2} \\[0pt]
    #3
}

\newcommand{\sectionItem}[5]{ % Specify four parameters
    % Name, Date, Role, Location, Description
    \textbf{#1} \hfill \textbf{#2} \\[0pt]
    \role{#3}{#4}{#5}    % Use the role command to specify the role and location
}

\linespread{1}
\setlist[itemize]{label=\textbullet, noitemsep, topsep=0pt, leftmargin=10pt}

% ----------------------------------------------------------------

\begin{document}
\begin{center}
    \textbf{\LARGE Ishan Deshpande}
\end{center}

\vspace{-10pt}

\begin{center}
    \textbf{\small (512) 721-8095 | \href{mailto:ishdeshpa@utexas.edu}{ishdeshpa@utexas.edu} | \href{http://ishdeshpa.com}{ishdeshpa.com} | \href{http://linkedin.com/in/ishdeshpa/}{linkedin.com/in/ishdeshpa/}}
\end{center}

\vspace{-15pt}

\sectionHeader{Education}       
\begin{flushleft}
    \textbf{\large UT Austin} \\
    \textbf{B.S. in Electrical and Computer Engineering} \hfill \textbf{May 2025} \\
    \textbf{M.S. in Electrical and Computer Engineering (Spec. Embedded Systems)} \hfill \textbf{May 2026} \\
    \textit{3.79 Undergraduate GPA} \\
    \textbf{Relevant Coursework:} Embedded Systems Lab, Compilers,Operating Systems, Digital Logic Design, ML/HW Codesign, Computer Architecture, Algorithms, Data Science Lab, Software Design \& Implementation I/II, Senior Design
\end{flushleft}

\vspace{-5pt}

\sectionHeader{Experience}
\begin{flushleft}
    \sectionItem{Tesla}{May 2024 - Aug 2024}{Body Controls Firmware Intern}{Palo Alto, CA}{
        \begin{itemize}
            \item Enhanced future firmware reliability by implementing platform-wide tooling PC-Lint for static code analysis
            \item Assessed non-volatile memory usage across all body control modules by introducing CAN logging for EEPROM usage
            \item Improved vehicle performance by fixing various bugs and implementing features related to sensor calibration and interfacing (mirror heaters, trailer brake sensors, overhead lights, door actuation)     
        \end{itemize}
    }
    
    \sectionItem{Garmin}{May 2023 - Aug 2023}{Flight Control Systems Firmware Intern}{Olathe, KS}{
        \begin{itemize}
            \item Accelerated aircraft control module testing by writing a Python SIL simulation for vertical navigation and altitude tracking
            \item Optimized flight control module performance by developing HIL unit tests in C within an RTOS environment
        \end{itemize}
    }

    \sectionItem{University of Texas at Austin}{Jan 2023 - Present}{Operating Systems TA}{Austin, TX}{
        \begin{itemize}
            \item Enhance understanding of OS concepts for 70+ students through effective instruction
            \item Improve project outcomes by debugging issues in User Programs, Virtual Memory, and Filesystems with PintOS       
        \end{itemize}
        \role{Embedded Systems Tutor}{}{
            \begin{itemize}
                \item Elevated students' understanding of embedded systems by teaching key concepts such as interrupts, DAC/ADC, finite state machines, and serial communication, resulting in improved project performance and comprehension
            \end{itemize}
        }
    }

    \sectionItem{EnergyX}{May 2022 - Aug 2022}{Automation and Controls Intern}{Austin, TX}{
        \begin{itemize}
            \item Optimized the lithium extraction process by programming and integrating Programmable Logic Controllers (PLCs) onto a control panel, enhancing operational efficiency
            \item Increased system reliability by designing, wiring, and rigorously testing a control panel to meet prototype specifications
        \end{itemize}
    }
\end{flushleft}

\vspace{-7.5pt}

\sectionHeader{Activities and Projects}
\begin{flushleft}
    \sectionItem{Longhorn Racing Solar Car}{Sept 2021 - Present}{Platform Team Lead}{}{
        \begin{itemize} 
            \item Improved robustness of on-vehicle firmware updates by developing a UART/CAN bootloader (custom linker script and flash memory driver)
            \item Increased remote testing accessibility by designing a 100W power supply PCB featuring USB-C Power Delivery input and 12V output
        \end{itemize}
        \role{Controls Team Lead}{}{
            \begin{itemize}
                \item Programmed in an RTOS context for our shared platform of ARM-based STM32 microcontrollers
                \item Interfaced with display, pedals, lights, and motor controller
                \item Managed a team of 12 through a traditional software workflow with Git/GitHub
            \end{itemize}
        }
    }
    \sectionItem{UT Systems and Machine Learning Research}{Oct 2024 - Present}{Undergraduate Research Assistant}{}{
        \begin{itemize}
            \item Developing a machine learning model to predict the energy consumption of a serverless computing platform
        \end{itemize}
    }
    \sectionItem{UT Embedded Systems Lab Design Competition - 3rd Place}{Apr 2024}{Embedded Software Engineer}{}{
        \begin{itemize}
            \item Collaborated with a team of 4 to develop a GameBoy emulator on a TM4C microcontroller
            \item Designed a circuit board in KiCAD and worked with protocols such as SPI/QSPI and 16-bit parallel
        \end{itemize}
    }
    \sectionItem{Data Science Lab Project}{Nov 2023 - Dec 2023}{ML Engineer}{}{
        \begin{itemize}    
            \item Stacked ML models to restore portraits on UT TACC’S Frontera supercomputer for the UT History Department
        \end{itemize}
    }
\end{flushleft}

\begin{flushleft}
    \rule{\textwidth}{1pt}
    \underline{Skills:} C/C++, KiCAD, Python, Verilog, Vitis HLS, Kernel Development, Application Development for RTOS, Linux, Git, Java \\
    \underline{Interests:} Table Tennis, Music Production, Piano
\end{flushleft}

\end{document}