\documentclass[12pt]{article}
\usepackage{geometry}
\geometry{a4paper, margin=1in}
\usepackage{setspace}
\setstretch{1.5}

\title{Statement of Purpose\\ \large UT Austin ECE Stage 2 Integrated Master’s Program}
\author{\small Ishan Deshpande}
\date{}

\begin{document}

\maketitle

I am applying to the Stage 2 Integrated Master’s program in the ACSES track because I want to work on embedded systems and operating systems design in the future. My undergraduate focus has been on Computer Architecture and Embedded Systems, so continuing with the ACSES track feels like the natural next step.

My favorite and most valuable coursework so far has been Embedded Systems Design Lab, Computer Architecture, Operating Systems, and Compilers. At the graduate level, I’ve taken courses like Hardware/ML Codesign with Dr. Erez and Unconventional Computation with Dr. Soloveichik, and I plan to take SysML with Dr. Yadwadkar and Human Signals with Dr. Thomaz next semester. These classes have pushed me to think at a higher level and shaped my interest in designing “hardware-aware” software, especially in areas like device drivers or kernel development.

Outside of class, I’ve gained a lot of experience through the UT Solar Vehicle Team, where I’ve worked on circuit board design and embedded software for an ARM-based RTOS on STM32 microcontrollers. Recently, I designed my own UART bootloader and CADded \& soldered a 100W power supply PCB for bench development. This team has given me a unique opportunity to work on fully custom embedded systems for an automotive application, which has been a big influence on my engineering interests as a graduate student. I hope that in both my future personal projects, research, and classwork, I can continue to work on similar projects that combine hardware and software design.

My previous firmware internships have significantly contributed to my growth as an engineer. At Garmin, I worked on Software-in-the-Loop testing, which deepened my understanding of simulation-based validation and requirements-driven development. Last year at Tesla, I developed features for control modules, gaining experience in a professional development environment by working on non-volatile memory logging and interfacing with external sensors across various microcontrollers. Internships like these are insightful for understanding how embedded systems are used in industry and have helped me develop a strong foundation in systems design, and I hope to build on these skills in the ACSES track.

In addition to industry work, I’m also interested in research. This semester, I joined the UT SysML research group under Dr. Yadwadkar, where I’ve been working on improving energy efficiency and performance in serverless computing platforms. This work has introduced me to interesting problems around how hardware use affects energy consumption, and I’ve enjoyed thinking through and implementing solutions on the cutting-edge of SysML research. My Senior Design project with Dr. Erez is also research-focused, where we’re analyzing trends in a computer architecture simulator to help researchers design more efficient systems. Both of these experiences have shaped my interest in impactful research and made me excited to continue in this field.

The Stage 2 Master’s program will help me build the skills I need to contribute to these kinds of challenges. UT’s ECE faculty has played a big part in shaping my education and goals. Professors like Dr. Yerraballi, Dr. Erez, Dr. Yadwadkar, and Dr. Valvano have been especially influential in how I think about engineering. I’d love to keep working with UT SysML and learning more about embedded and low-level systems as I continue in this program.

In short, the ACSES track is the next step in my academic and professional journey. It will help me grow as an engineer and work toward solving important problems, both in embedded systems and beyond.

\end{document}
